\documentclass[main.tex]{subfiles}

\begin{document}
\section{Úvod}

\subsection{Obecné informace}
\begin{frame}[fragile]{Obecné informace}
  \begin{itemize}
    \item Textové prostředí (\textit{shell}) se skriptovacím jazykem
    \item Vyvíjené společností Microsoft od roku 2006
      \begin{itemize}
        \item Podrobná oficiální dokumentace
      \end{itemize}
    \item Založeno na platformě .NET (předtím Framework, nyní Core)
    \item V posledních letech se rozšiřuje na ostatní systémy
      \begin{itemize}
        \item Linux, Mac OS, ARM (Raspberry Pi)
      \end{itemize}
  \end{itemize}
\end{frame}

\subsection{Desktop vs. Core}
\begin{frame}{Windows Powershell}
  \begin{columns}
    \begin{column}{0.5\textwidth}
      \begin{itemize}
        \item Poslední verze 5.1
        \item Založený na .NET Framework
        \item Dostupný pouze na Windows
        \item Na systému již předinstalovaný
        \item Vývoj byl ukončen - nyní se pouze udržuje
          \begin{itemize}
            \item Stále se vydávají opravy chyb a bezpečnostní záplaty
          \end{itemize}
      \end{itemize}
    \end{column}
    \begin{column}{0.5\textwidth}
      \begin{center}
      \includegraphics[width=0.8\textwidth]{img/logo2.png}
      \end{center}
    \end{column}
  \end{columns}
    \end{frame}
    \begin{frame}[allowframebreaks]{Powershell Core}
     \begin{columns}
       \begin{column}{0.5\textwidth}
      \begin{itemize}
     \item Uveden v srpnu 2016 s verzí 6.0
     \item Založený na .NET Core Runtime
     \item Poslední verze 7.0.0 (experimentální)
       \item Open-source
         \begin{itemize}
           \item Instalace z balíčku na \href{https://github.com/PowerShell/PowerShell}{Githubu} (zde je popsán podrobný návod pro~všechny systémy)
         \end{itemize}
     \end{itemize}
       \end{column}
       \begin{column}{0.5\textwidth}
         \begin{center}
          \includegraphics[width=0.8\textwidth]{img/logo.png}
         \end{center}
       \end{column}
     \end{columns}
         \begin{center}
\includegraphics[width=1\textwidth]{img/install_distros.png}
         \end{center}
         \begin{itemize}
       \item Multiplatformní
       \item Velká část příkazů z verze 5.1 kvůli kompatibilitě zatím chybí
     \end{itemize}
\end{frame}

\subsection{Instalace}
\begin{frame}[allowframebreaks]{Instalace na Windows}
  Z \href{https://github.com/PowerShell/PowerShell/releases}{Githubového repozitáře} Powershellu (sekce Assets v Releases nebo Get Powershell v README) stáhneme instalační program pro náš systém.\\

  \includegraphics[width=1\textwidth,frame]{img/assetLink.png}\\

  Při instalaci klikáme na tlačítko Next, dokud se neobjeví tlačítko Install, na které také klikneme.


  \begin{center}
  \includegraphics[width=0.85\textwidth,frame]{img/wizard.png}
  \end{center}
\end{frame}
\begin{frame}{Instalace na Windows}
  \begin{center}
  \includegraphics[width=0.80\textwidth,frame]{img/wizard2.png}
  \end{center}
\end{frame}

\begin{frame}{Instalace na Windows}
  \begin{center}
  \includegraphics[width=0.80\textwidth,frame]{img/wizard3.png}
  \end{center}
\end{frame}

\begin{frame}{Instalace na Windows}
  \begin{center}
  Klikneme na tlačítko Finish.\\
  \end{center}
  \begin{center}

  \includegraphics[width=0.70\textwidth,frame]{img/wizard4.png}
  \end{center}
\end{frame}
\begin{frame}{Instalace na Windows}
  \begin{center}
    \includegraphics[width=1\textwidth,frame]{img/wizard5.png}
  \end{center}
\end{frame}


\begin{frame}[fragile,allowframebreaks]{Instalace na Arch Linux}
\begin{itemize}
  \item Neoficiální podpora ze strany uživatelské komunity
  \item Dostupné jako balíček \texttt{powershell-bin} v Arch User Repository
  \item Instaluje se pomocí správce balíčků, který podporuje AUR
\end{itemize}
        \begin{minted}[bgcolor=bg1]{shell-session}
[blanche@arch ~]$ yay -Sy powershell-bin
[sudo] password for blanche:

(1/4) installing liburcu
(2/4) installing numactl
(3/4) installing lttng-ust
(4/4) installing openssl-1.0

Total Installed Size:  151.00 MiB

:: Proceed with installation? [Y/n]
(1/1) Arming ConditionNeedsUpdate...
\end{minted}
\framebreak
\begin{minted}[bgcolor=bg1]{shell-session}
[blanche@arch ~]$ pwsh
PowerShell 7.0.0
Copyright (c) Microsoft Corporation. All rights reserved.

https://aka.ms/powershell
Type 'help' to get help.\end{minted}
\vspace{-13mm}
\begin{minted}[bgcolor=bg1]{ps1con}
PS /home/blanche> Write-Output "Hello, PowerShell!"
Hello, PowerShell!
\end{minted}
\end{frame}

\begin{frame}{Windows Terminal}
  \begin{columns}
    \begin{column}{0.45\textwidth}
 \begin{itemize}
  \item Představen v květnu 2019
  \item Dostupný ve Windows Store
  \item Open-source
  \item Moderní design konzole pro Windows 10
  \item Umožňuje používat více panelů
    \begin{itemize}
      \item V každém z nich lze spustit klasický CMD, PowerShell, nebo třeba linuxový terminál (možnost rozšíření)
    \end{itemize}
  \item Podpora ligatury (slitek) kódu
\end{itemize}
    \end{column}
    \begin{column}{0.55\textwidth}
      \includegraphics[width=\textwidth]{img/term.png}
    \end{column}
  \end{columns}
\end{frame}
\subsection{Vlastnosti jazyka}
\begin{frame}[allowframebreaks]{Popis jazyka}
  \begin{itemize}
    \item Přípona \texttt{.ps1}
    \item Objektově orientovaný se vším všudy
      \begin{itemize}
        \item Výstupy příkazů jsou (ve většině případů) také objekty
        \item Tím se liší například od klasického unixového shellu, kde jsou příkazy schopny vracet pouze text
      \end{itemize}
    \item Příkazy mohou, ale nemusí být zakončeny středníky (v případě, že chceme dostat více příkazů na jeden řádek)
      \begin{itemize}
        \item Stačí je oddělovat koncem řádku (\textit{newline})
      \end{itemize}
    \item Možnost spouštět skripty samostatně nebo psát příkazy do interaktivní konzole
    \item Načítání modulů (externí balíčky příkazů) - soubory \texttt{.psm1} nebo \texttt{.dll}
  \end{itemize}
      \framebreak
      \begin{itemize}
    \item \textbf{Pipeline} (roury) - přesměrování vstupu mezi příkazy
    \item \textbf{Cmdlet} - příkaz ve formátu \texttt{Sloveso-Podstatné jméno}
      \begin{itemize}
        \item např. \texttt{Get-Command}, \texttt{Write-Output}, \texttt{Add-Content}
      \end{itemize}
    \item Case-insensitive (nezáleží na velikosti písmen)
      \begin{itemize}
        \item \texttt{Get-Help} znamená to samé jako \texttt{get-help}
      \end{itemize}
    \item Podpora \textbf{aliasů}
    \item Možnost kombinace s unixovými příkazy (na které buď existuje alias pro cmdlet - \texttt{cat, pwd, echo}, nebo jsou přímou součástí systému - \texttt{lolcat, cmatrix, cowsay})
      \begin{itemize}
        \item Nedoporučuje se, skript pak nemusí být spustitelný na jiných systémech
      \end{itemize}

  \end{itemize} 
\end{frame}


\end{document}
