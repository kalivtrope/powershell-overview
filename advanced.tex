\documentclass[main.tex]{subfiles}

\begin{document}
\section{Pokročilé}

\subsection{Formuláře}
\begin{frame}[fragile,allowframebreaks]{Formuláře}
\begin{itemize}
  \item Tvorba grafických prostředí možná pouze pro Windows
  \item Založeno na platformě .NET Framework a její knihovně tříd Forms
  \item Nelze s ním pracovat v Powershell Core
  \item Ovládací prvky:
    \begin{itemize}
      \item Tlačítka: \texttt{Button, CheckBox, RadioButton}
      \item Textový vstup: \texttt{TextBox}
      \item Popis: \texttt{Label}
      \item Obrázek: \texttt{PictureBox}
      \item Ukazatel průběhu: \texttt{ProgressBar}
      \item Rozevírací nabídka: \texttt{ComboBox}
      \item Seskupení několika prvků: \texttt{GroupBox}
    \end{itemize}
\end{itemize}
\begin{center}
  \textbf{Ukázka použití ovládacích prvků.}

  Zdrojový kód dostupný \href{https://gist.github.com/kalivtrope/1c8edaace6c65225010a6791fcef3fa6}{zde}.\\[2mm]

  \includegraphics[width=0.65\textwidth]{img/sample.png}
\end{center}
\framebreak
\textbf{Atributy ovládacích prvků}
\begin{itemize}
  \item K textovému obsahu komponent se přistupuje v atributu \textbf{Text}
    \begin{itemize}
      \item U samotného formuláře tato vlastnost určuje název okna
      \item Mimojiné jde měnit i \textbf{Font} textu, například
        \mintinline{powershell}|$label.Font = 'Segoe UI font,12'|
    \end{itemize}
  \item Velikost prvku je uložena v atributu \textbf{Size} (a také \textbf{Width,Height})
    \begin{itemize}
      \item Dá se nastavit dvěma způsoby:
        \begin{enumerate}
          \item Pomocí objektů v \texttt{System.Drawing} (třeba přidat přes \texttt{Add-Type}) \mintinline{powershell}|$form.Size = New-Object System.Drawing.Size($width,$height)|
          \item Deklarací v řetězci\\ \mintinline{powershell}|$form.Size = '{0},{1}' -f $width,$height|
        \end{enumerate}
      \item Také se může dopočítat automaticky (v závislosti na rozměru obsahu) nastavením booleovské vlastnosti \textbf{AutoSize}
    \end{itemize}
  \item Pozice komponenty se definuje v členu \textbf{Location}
    \begin{itemize}
      \item Udává se v pixelech
      \item Vzdálenost se měří od levého horního rohu okna
      \item K přiřazení se může využít \texttt{System.Drawing.Point} nebo opět řetězec
    \end{itemize}
\end{itemize}
  \framebreak
    \begin{itemize}
  \item \textit{Atributy specifické pro formulář}
      \begin{itemize}
        \item Vlastnost \textbf{StartPosition} umožňuje zarovnání formuláře na obrazovce, nejčastěji na střed:\\
          \mintinline{powershell}|$form.StartPosition = 'CenterScreen'|
        \item Boolean \textbf{TopMost} udává, zda se formulář bude zobrazovat nad všemi ostatními (i při přepnutí na jiné okno)
        \item Do kolekce \textbf{Controls} je třeba přidat veškeré ovládací prvky, které chceme ve formuláři zobrazit
          \begin{enumerate}
            \item \mintinline{powershell}|$form.Controls.AddRange(@($label, $userInput, $genButton))|
            \item \begin{minted}{powershell}
$form.controls.Add($label)
$form.Controls.Add($userInput)
$form.Controls.Add($genButton)
              \end{minted}
          \end{enumerate}
        \item Ohraničení okna se nastaví v atributu \textbf{FormBorderStyle}
          \begin{itemize}
            \item Hodnota \mintinline{powershell}|'FixedDialog'| zabrání změně velikosti okna
          \end{itemize}
  \item Po veškeré konfiguraci a přidání do formuláře dojde k jeho zobrazení metodou \textbf{ShowDialog}
    \end{itemize}
\end{itemize}
\framebreak
\textbf{Zachytávání událostí}
\begin{itemize}
      \item K informacím o události lze v kontrolních funkcích opět přistoupit v automatické proměnné \texttt{\$\_}
      \item Ovládací funkce musí být typu \texttt{ScriptBlock}, proto pokud jsme původně deklarovali samostatnou funkci, je třeba ji při přidání obalit do \texttt{\{ \}}
  \item \textit{Kliknutí myši}
    \begin{itemize}
      \item Ovládací funkce se přidává pomocí metody \textbf{Add\_Click}
        \mintinline{powershell}|$genButton.Add_Click({Shuffle})|
    \end{itemize}
  \item \textit{Stisknutí klávesy}
    \begin{itemize}
      \item Je třeba zapnout booleovský atribut formuláře \textbf{KeyPreview}
        \begin{itemize}
          \item Ten umožní formuláři obdržet jednotlivé klávesy
          \item Jinak je dostává pouze aktivní komponent (např. textové pole, do kterého se píše)
        \end{itemize}
        \framebreak
      \item Přidává se metodou \textbf{Add\_KeyDown}:
\begin{minted}{powershell}
$form.Add_KeyDown({
  if($_.KeyCode -eq 'Enter'){
  Shuffle
  }
})
\end{minted}
\end{itemize}
\end{itemize}
\vspace{3mm}
\begin{itemize}
  \item Pro přístup k funkcím Forms je třeba načtení pomocí \texttt{Add-Type}
\begin{minted}{powershell}
Add-Type -AssemblyName System.Windows.Forms
\end{minted}
\end{itemize}
\end{frame}

\begin{frame}[fragile,allowframebreaks]{Ukázka formulářů}
  \begin{center}
    \textbf{Aplikace, která náhodně zpřehází písmenka ve vstupním řetězci.}\\
  \vspace{3mm}
  \includegraphics[width=\textwidth,frame]{img/form.png}
  \end{center}
\begin{minted}[linenos,fontsize=\footnotesize]{powershell}
Add-Type -AssemblyName System.Windows.Forms
Add-Type -AssemblyName System.Drawing

$fontFam = 'Segoe UI font'
$fontSize = 12

# Vlastnosti formuláře
$form = New-Object System.Windows.Forms.Form
$form.Text = 'WordMixer'
$form.Size = New-Object System.Drawing.Size(600,200)
$form.StartPosition = 'CenterScreen'
$form.TopMost = $true

# Popisek textového pole
$label = New-Object System.Windows.Forms.Label
$label.Font = $fontFam + ',' + $fontSize
$label.Location = New-Object System.Drawing.Point(10,
                    ($form.Size.Height / 4))
$label.Width = 150
$label.Height = $fontSize * 3
$label.Text = 'Zadejte text: '

# Textové pole pro uživatelský vstup
$userInput = New-Object System.Windows.Forms.TextBox
$userLocX = $label.Location.X + $label.Width
$userInput.Location = New-Object System.Drawing.Point($userLocX,
                        $label.Location.Y)
$userInput.Font = $fontFam + ',' + $fontSize
$userInput.Width = 200

function Shuffle {
  $userInput.Text = ($userInput.Text -split '' |
                       Sort-Object {Get-Random}) -join ''
}

# Tlačítko pro generování
$genButton = New-Object System.Windows.Forms.Button
$genButton.Text = "Protřepat, nemíchat!"
$genButton.AutoSize = $true
$buttonLocX = $userInput.Location.X + $userInput.Width + 5
$genButton.Location = New-Object System.Drawing.Point($buttonLocX,
                        $label.Location.Y)
$genButton.Add_Click({Shuffle})

$form.KeyPreview = $true
# Možnost stisknout také Enter pro novou generaci
$form.Add_KeyDown({
  if($_.KeyCode -eq 'Enter'){
    Shuffle
  }
})

$form.controls.Add($label)
$form.Controls.Add($userInput)
$form.Controls.Add($genButton)
$form.ShowDialog()
\end{minted}
\end{frame}

\subsection{Nástroje}
\begin{frame}[allowframebreaks]{Nástroje pro tvorbu GUI}
  \textbf{Powershell Studio}
\begin{itemize}
  \item Vzhledově podobné Visual Studiu
  \item Obsahuje grafický návrhář a debugger
  \item Převod skriptů na spustitelné (\texttt{.exe}) soubory
  \item Vytváření instalačních (\texttt{.msi}) skriptů
  \item Tvorba modulů
  \item Integrovaná konzole
  \item Monitorování výkonu a využití paměti
  \item Podpora klasického Powershellu a Powershell Core
  \item Placené (\textbf{\$399})
\end{itemize}
\begin{center}
\textbf{Powershell Studio}\\
\vspace{3mm}
\includegraphics[height=0.72\textheight,frame]{img/pstudio-main.png}
\end{center}
\framebreak
\textbf{POSHGUI}
\begin{itemize}
  \item Online nástroj pro výrobu grafických prostředí v Powershellu
  \item Obsahuje repozitář uživatelských projektů
  \item Možnost grafické tvorby Cmdletů
  \item Zdarma
\end{itemize}
\framebreak
\begin{center}
\textbf{POSHGUI}\\
\vspace{3mm}
\includegraphics[width=1.2\textheight,frame]{img/posh.png}
\end{center}
\end{frame}


\subsection{Remoting}
\begin{frame}{Remoting}
  Spouštění kódu na vzdálených zařízeních (dostupných po síti).\\

  \textbf{Windows Powershell}
  \begin{itemize}
    \item Pouze pro Windows
    \item Spojení přes WinRM (Windows Remote Management)
  \end{itemize}
  \textbf{Powershell Core}
  \begin{itemize}
    \item Multiplatformní
    \item Komunikace pomocí SSH (Secure Shell)
    \item Snazší konfigurace zabezpečení (netřeba nastavovat SSL/TLS certifikáty kvůli HTTPS)
  \end{itemize}
\end{frame}
\begin{frame}[fragile,allowframebreaks]{Konfigurace SSH na Windows}
  OpenSSH - Open-source implementace SSH\\

  \textbf{Instalace OpenSSH z Powershellu (jako Administrátor)}
\begin{enumerate}
  \item Zjištění dostupnosti balíčků
\begin{minted}{powershell}
Get-WindowsCapability -Online | ? Name -like 'OpenSSH*'
\end{minted}
Mělo by vypsat (pokud již není nainstalované):
\begin{verbatim}
Name  : OpenSSH.Client~~~~0.0.1.0
State : NotPresent
Name  : OpenSSH.Server~~~~0.0.1.0
State : NotPresent
\end{verbatim}
\framebreak
\item Instalace klienta a serveru
\begin{minted}[fontsize=\small]{powershell}
Add-WindowsCapability -Online -Name OpenSSH.Client~~~~0.0.1.0
Add-WindowsCapability -Online -Name OpenSSH.Server~~~~0.0.1.0
\end{minted}
Oba příkazy by měly vrátit:
\begin{verbatim}
Path          :
Online        : True
RestartNeeded : False
\end{verbatim}
\framebreak
\item Automatické spouštění ssh
\begin{minted}{powershell}
Start-Service sshd
Set-Service -Name sshd -StartupType 'Automatic'
\end{minted}
\item Konfigurace firewallu pro server
\begin{minted}[breaklines]{powershell}
New-NetFirewallRule -Name sshd -DisplayName 'OpenSSH Server (sshd)' -Enabled True -Direction Inbound -Protocol TCP -Action Allow -LocalPort 22
  \end{minted}
\item Úprava konfiguračního souboru \texttt{\$env:ProgramData\textbackslash ssh\textbackslash \textbf{sshd\_config}} (např. přes \texttt{notepad.exe} pořád v Powershellu) - přidání dvou řádků
  \suspend{enumerate}
      \scriptsize\begin{verbatim}
# 1) Povolí ověřování heslem
PasswordAuthentication Yes

# 2) Vytvoří subsystém (sadu příkazů, přes kterou se klient bude
# pohodlně připojovat na server) pro Powershell.
# Počítá se s výchozí cestou instalace pwsh
# (jinak je třeba dosadit vlastní)
Subsystem powershell c:/progra~1/powershell/7/pwsh.exe -sshs -NoLogo -NoProfile\end{verbatim}
\normalsize
\resume{enumerate}
\item Restartování služby \texttt{sshd}
\begin{minted}{powershell}
Restart-Service sshd
\end{minted}
  \end{enumerate}
\end{frame}

\begin{frame}[fragile,allowframebreaks]{Konfigurace SSH na Linuxu}
\begin{enumerate}
  \item OpenSSH je dostupné ve správci balíčků jako \texttt{openssh} nebo \texttt{openssh-server + openssh-client} (liší se u různých distribucí)
    \begin{itemize}
      \item Příklad pro Ubuntu:
    \begin{verbatim}
sudo apt install openssh-client
sudo apt install openssh-server
    \end{verbatim}
\end{itemize}
\item Úprava konfiguračního souboru \texttt{/etc/ssh/sshd\_config} - přidání dvou řádků
  \suspend{enumerate}
  \begin{verbatim}
PasswordAuthentication Yes
# Umístění powershellu lze zjistit příkazem "whereis pwsh"
Subsystem powershell /usr/bin/pwsh -sshs -NoLogo -NoProfile\end{verbatim}

\resume{enumerate}

\item Restartování služby \texttt{sshd}
  \begin{itemize}
    \item Např. přes \texttt{sudo systemctl restart sshd}
    \item Permanentní zapnutí přes: \texttt{sudo systemctl enable sshd}
  \end{itemize}
\end{enumerate}
\end{frame}

\end{document}
