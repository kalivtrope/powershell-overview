\documentclass[main.tex]{subfiles}

\begin{document}
\section{Syntax}
\subsection{Proměnné}
\begin{frame}[fragile]{Proměnné}
\begin{itemize}
  \item Slouží k ukládání informací do paměti
  \item Zapisují se s \$ na začátku
  \item Na velikosti písmen při zápisu nezáleží (\textit{Case-insensitive})
  \item K přiřazení hodnoty se používá = (\textit{inicializace})
  \item Hodnota se ale nemusí přiřadit, pokud proměnnou pouze nazveme (\textit{deklarace})
  \item Typ proměnné rovněž nemusí být upřesněn, ale dá se zapsat v [ ] před~její název
    \begin{itemize}
      \item Případně se tak proměnná dá přetypovat
    \end{itemize}
\end{itemize}

\vspace{5mm}
\begin{columns}
    \begin{column}{0.4\textwidth}
        \textbf{Typovaná deklarace}
        \large\mint{powershell}|[string]$greetings|
    \end{column}
    \begin{column}{0.5\textwidth}
        \textbf{Netypovaná inicializace}
        \large\mint{powershell}|$greetings = "Ahoj světe!"|
    \end{column}
\end{columns}
\end{frame}
\begin{frame}[fragile,allowframebreaks]{Přehled běžně používaných typů}
  \begin{itemize}
    \item Přejaté z platformy .NET
    \item K zjištění typu proměnné se používá zřetězení metody a členu \mintinline{powershell}|.GetType().FullName|
  \end{itemize}
  \begin{center}
    \vspace{-4mm}
  \renewcommand{\arraystretch}{1.2}
    \begin{tabular}{|c|c|c|}
   \hline
   \textbf{Název} & \textbf{Popis} & \textbf{Příklad} \\
   \hline\hline
   {\color{typecolor}\texttt{[char]}} & Jeden znak (s podporou Unicode) & {\color{codered}\DejaSans '☺'} \\
   {\color{typecolor}\texttt{[string]}} & Textový řetězec - posloupnost znaků & {\color{codered}\texttt{"red"}}\\
   {\color{typecolor}\texttt{[bool]}} & Logická hodnota & {\color{varcolor}\texttt{\$true}}, {\color{varcolor}\texttt{\$false}} \\
   {\color{typecolor}\texttt{[int]}} & Celé číslo (32-bit) & \texttt{42}\\
   {\color{typecolor}\texttt{[long]}} & Celé číslo (64-bit) & \texttt{1982989219753}\\
   {\color{typecolor}\texttt{[float]}} & Desetinné číslo (přesnost \textasciitilde{}7 řádů) & \texttt{3.141593}\\
   {\color{typecolor}\texttt{[double]}} & Desetinné číslo (přesnost \textasciitilde{}15 řádů) & \texttt{3.14159265359}\\
   {\color{typecolor}\texttt{[decimal]}} & Desetinné číslo (přesnost \textasciitilde{}28 řádů) & \texttt{0.1234567891011}\\
   {\color{typecolor}\texttt{[DateTime]}} & Datum a čas & {\color{cmdletcolor}\texttt{Get-Date}} \\
   {\color{typecolor}\texttt{[Object[]]}} & Kolekce prvků  & \texttt{@(}{\color{codered}\texttt{'a'}}\texttt{,}{\color{codered}\texttt{"B"}}\texttt{, 2.7)} \\
   \hline
  \end{tabular}
  \end{center}
  \renewcommand{\arraystretch}{1}
\end{frame}

\begin{frame}[fragile]{Komentáře}
   \begin{itemize}
    \item Kód, který není vyhodnocován jako příkaz
    \item Používá se k dokumentaci kódu nebo (dočasnému) odebrání některé funkcionality
    \item Jednořádkový (tj.\ do konce řádku) začíná s \texttt{\textit{\#}}
    \item Víceřádkový je pak obsažen v \texttt{\textit{\textless\# \#\textgreater}}
  \end{itemize}
  \vspace{5mm}
  \begin{columns}[t]
    \begin{column}{0.42\textwidth}
     \textbf{Jeden řádek}
     \begin{minted}{powershell}
# Pozdraví svět
Write-Output "Ahoj světe!"
     \end{minted}
    \end{column}
    \begin{column}{0.58\textwidth}
      \textbf{Více řádků}
     \begin{minted}{powershell}
<#
  Toto je velice důležitá proměnná.
  Už jsem sice zapomněl, co dělá,
  ale pokud se odstraní,
  program začne náhodně padat.
#>
$x = 42
      \end{minted}
    \end{column}
  \end{columns}
\end{frame}

\subsection{Cmdlety}
\begin{frame}[allowframebreaks]{Cmdlety}
  \begin{itemize}
    \item Kompilovaný kód C\# (nebo čehokoliv, co podporuje .NET)
    \item Zkratka pro \texttt{"Command let"}
    \item Příkaz ve formátu \texttt{Sloveso-PodstatnéJméno}
      \begin{itemize}
        \item \textit{Sloveso} vyjadřuje prováděnou operaci
        \item \textit{Podstatným jménem} (standardně v jednotném čísle) označíme objekty, na kterých operaci vykonáme
      \end{itemize}
   \item Kromě tohoto pravidla by se také měla používat standardní slovesa
     \begin{itemize}
       \item Dají se zobrazit příkazem \texttt{Get-Verb}
     \end{itemize}
  \item Na část z nich se lze odkázat přes aliasy
    \begin{itemize}
      \item Definované zvlášť pro slovesa a podstatná jména, kombinují se dohromady
    \end{itemize}
  \end{itemize}
  \framebreak
  \textbf{Výběr některých běžně používaných sloves v Cmdletech}
  \begin{center}
  \begin{tabular}{|l|c|l|c|}
    \hline
    \textbf{Název} & \textbf{Alias} & \textbf{Příklad} & \textbf{Alias příkladu} \\
    \hline
    Add & a & \texttt{Add-Content} & \\
    \hline
    Clear & cl & \texttt{Clear-History} & clhy \\
    \hline
    Copy & cp & \texttt{Copy-Item} & cpi \\
    \hline
    Get & g & \texttt{Get-Process} & gps \\
    \hline
    Find & fd & \texttt{Find-Module} & \\
    \hline
    Format & f & \texttt{Format-Table} & ft\\
    \hline
    Join & j & \texttt{Join-Path} & \\
    \hline
    Move & m & \texttt{Move-Item} & mi \\
    \hline
    New & n & \texttt{New-Alias} & nal \\
    \hline
    Remove & r & \texttt{Remove-Variable} & rv \\
    \hline
    Set & s & \texttt{Set-Location} & sl\\
    \hline
    Show & sh & \texttt{Show-Markdown} & \\
    \hline
    Split & sl & \texttt{Split-Path} & \\
    \hline
  \end{tabular}
\end{center}
\end{frame}

\subsection{Operátory}
\begin{frame}[fragile,allowframebreaks]{Aritmetika}
  \textit{Sčítání}
  \begin{itemize}
    \item Podporováno mezi čísly, řetězci, poli a slovníky
  \end{itemize}
  \vspace{5mm}
  \resizebox{\textwidth}{!}{%
  \begin{tabular}{|m{6.8cm}|m{6cm}|}
    \hline
    \textbf{Výraz} & \textbf{Výsledek} \\
    \hline
    \mintinline{powershell}|41 + 1| & \mintinline{powershell}|42| \\
    \mintinline{powershell}|"abc" + "def"| & \mintinline{powershell}|"abcdef"| \\
    \mintinline{powershell}|@(1, "jedna") + @(2.0, "dva")| & \mintinline{powershell}|@(1, "jedna", 2.0, "dva")| \\
    \mintinline{powershell}|@{'A' = "Alfa"} + @{'B' = "Bravo"}| & \mintinline{powershell}|@{'A' = "Alfa"; 'B' = "Bravo"}| \\
    \hline
      \end{tabular}}

   \framebreak
   \textit{Odčítání}
   \begin{itemize}
     \item Možné pouze u čísel
     \item Binární (mezi dvěma čísly): \mintinline{powershell}|18 - 8|
     \item Unární (převrácení hodnoty): \mintinline{powershell}|-7|
   \end{itemize}
  \vspace{3mm}
   \textit{Násobení}
   \begin{itemize}
     \item Definováno v rámci čísel a polí
   \end{itemize}
  \vspace{3mm}
  \resizebox{\textwidth}{!}{%
  \begin{tabular}{|m{6.8cm}|m{6cm}|}
    \hline
    \textbf{Výraz} & \textbf{Výsledek} \\
    \hline
    \mintinline{powershell}|6 * 7| & \mintinline{powershell}|42| \\
    \mintinline{powershell}|"abc" * 5| & \mintinline{powershell}|"abcabcabcabcabc"| \\
    \mintinline{powershell}|@(7, 1) * 3| & \mintinline{powershell}|@(7, 1, 7, 1, 7, 1)| \\
    \hline
      \end{tabular}}
   \framebreak

   \textit{Dělení}
   \begin{itemize}
     \item Pokud není určený typ, výsledek se dle potřeby dokáže převést buď na~celé, nebo desetinné číslo
     \item \mintinline{powershell}|(6/3).GetType().FullName| $\to$ {\color{codered}\texttt{ System.Int32}} (název v .NET)
     \item \mintinline{powershell}|(6/4).GetType().FullName| $\to$ {\color{codered}\texttt{ System.Double}}
   \end{itemize}
  \vspace{5mm}
   \textit{Modulo}
   \begin{itemize}
     \item Vrací zbytek po dělení dvou (klidně i desetinných) čísel
     \item \mintinline{powershell}|16 % 5| $\to$ \texttt{ 1}
     \item \mintinline{powershell}|5.3 % 2| $\to$ \texttt{ 1.3}
   \end{itemize}
   \vspace{3mm}
   \textit{Ostatní matematické funkce}
  \begin{itemize}
    \item Dostupné v .NET třídě \texttt{Math}
    \item \mintinline{powershell}|[Math]::floor([Math]::pi) # 3|
  \end{itemize}
   \framebreak

   \textit{Aritmetika v rámci bitů} (bitwise arithmetic)
 \begin{itemize}
   \item Pracuje s každým bitem čísla zvlášť
   \item Číslo lze zapsat v binární formě s prefixem \texttt{0b}, například
     \mintinline{powershell}|0b111| $\to$ \texttt{ 7}
 \end{itemize}
  \resizebox{1\textwidth}{!}{%
  \begin{tabular}{|c|c|c|c|}
    \hline
    \textbf{Název operace} & \textbf{Operátor} & \textbf{Výsledek na bitech $a(,b)$} & \textbf{Příklad} \\
    \hline
    Binární součin & {\color{paramcolor}\texttt{-band}} & $a\cdot b$ & \begin{tabular}{r} \texttt{0b1100} \\ \texttt{AND 0b1010}\\ \hline \texttt{0b1000} \end{tabular} \\ \hline
    Binární součet & {\color{paramcolor}\texttt{-bor}} & $a+ b$ & \begin{tabular}{r} \texttt{0b1100} \\ \texttt{OR 0b1010}\\ \hline \texttt{0b1110} \end{tabular} \\ \hline
    Exkluzivní součet & {\color{paramcolor}\texttt{-bxor}} & $a\not =b$ & \begin{tabular}{r} \texttt{0b1100} \\ \texttt{XOR 0b1010}\\ \hline \texttt{0b0110} \end{tabular} \\ \hline
    Binární negace & {\color{paramcolor}\texttt{-bnot}} & $!a$ &  \begin{tabular}{c|c} $a$ & \texttt{0b10101110} \\ \hline $!a$ & \texttt{0b01010001} \end{tabular}\\ \hline
    Posun vlevo & {\color{paramcolor}\texttt{-shl}} &  & \begin{tabular}{c|c} $a$  & \texttt{0b10101110} \\ \hline $a << 3$ & \texttt{0b01110000} \end{tabular}  \\ \hline
    Posun vpravo & {\color{paramcolor}\texttt{-shr}} &  & \begin{tabular}{c|c} $a$  & \texttt{0b10101110} \\ \hline $a >> 3$ & \texttt{0b00010101} \end{tabular}  \\ \hline
      \end{tabular}}
  \end{frame}

\end{document}
