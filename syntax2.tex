\documentclass[main.tex]{subfiles}

\begin{document}

\begin{frame}[fragile]{Porovnávání}
  \begin{center}
  \resizebox{\textwidth}{!}{%
  \begin{tabular}{|c m{1.7cm} m{4.3cm} m{4.2cm}|}
    \hline
    \textbf{Název operace} & \textbf{Operátor} & \textbf{Pravdivý výraz} & \textbf{Nepravdivý výraz} \\
    \hline
  Rovná se & \mintinline{powershell}|-eq| & \mintinline{powershell}|"abc" -eq "ABC"| & \mintinline{powershell}|5 -eq 8| \\
  Nerovná se & \mintinline{powershell}|-ne| & \mintinline{powershell}|"abc" -ne "def"| & \mintinline{powershell}|42 -ne 42| \\
  Větší než & \mintinline{powershell}|-gt| & \mintinline{powershell}|9 -gt 5| & \mintinline{powershell}|5 -gt 5| \\
  Větší nebo rovno & \mintinline{powershell}|-ge| & \mintinline{powershell}|8 -ge 8| & \mintinline{powershell}|7 -ge 12| \\
  Menší než & \mintinline{powershell}|-lt| & \mintinline{powershell}|12 -lt 24| & \mintinline{powershell}|24 -lt 12| \\
  Menší nebo rovno & \mintinline{powershell}|-le| & \mintinline{powershell}|'a' -le "z"| & \mintinline{powershell}|'f' -le 'd'| \\
  Wildcard match & \mintinline{powershell}|-like| & \mintinline{powershell}|"ABCDEF" -like "a*"| & \mintinline{powershell}|"ABCDEF" -like "b*"| \\
  Regex match & \mintinline{powershell}|-match| & \mintinline{powershell}|"ABCDEF" -match "BC"| & \mintinline{powershell}|"ABCDEF" -match "^B*"| \\
  Přítomnost &  \mintinline{powershell}|-in| & \mintinline{powershell}|"a" -in @('a',"b","c")| & \mintinline{powershell}|"a" -in @("b","c")| \\
  Typová rovnost & \mintinline{powershell}|-is| & \mintinline{powershell}|'a' -is [string]| & \mintinline{powershell}|5 -is [char]| \\
  \hline
  \end{tabular}}
  \end{center}
\end{frame}

\subsection{Větvení}%
\begin{frame}[fragile,allowframebreaks]{Podmínky}
  Větvení programu na základě pravdivosti logických výrazů.\\

  \textbf{Příkaz If/Else}
    \begin{itemize}
      \item Skládá se z částí \textit{If, ElseIf} a \textit{Else}
      \item \textit{ElseIf} a \textit{Else} jsou nepovinné (ale jejich pořadí se musí dodržovat)
      \item \textit{ElseIf} může být zopakováno neomezeně
      \item Jakmile se program do jedné z větví zanoří, po vykonání příkazů skočí za konec \textit{Else} (nebo za poslední část v sérii)
    \end{itemize}
    \framebreak
        \textbf{Obecný zápis}
\begin{center}
  
\begin{minted}[mathescape,fontsize=\small]{powershell}
if( <podmínka1> ) {
  <příkazy>
}
elseif( <podmínka2> ) {
  <příkazy>
}
# $\ldots$
elseif( <podmínkaN> ) {
  <příkazy>
}
else {
  <příkazy>
}
    \end{minted}
\end{center}
 \framebreak
     \textbf{Příklad: BMI kalkulačka}
     \begin{center}
    \begin{minted}[fontsize=\small]{powershell}
$weight = Read-Host "Zadejte hmotnost (v kg)"
$height = (Read-Host "Zadejte výšku (v cm)") / 100

$bmi = $weight / ($height * $height)

Write-Host "BMI:" $bmi

if ($bmi -le 16.5) {
  Write-Host -ForegroundColor Blue "Těžká podvýživa." 
}
elseif ($bmi -le 18.5) {
  Write-Host -ForegroundColor Cyan "Podváha."
}
elseif ($bmi -le 25) {
  Write-Host -ForegroundColor Green "Ideální (zdravá) váha."
}
elseif ($bmi -le 30) {
  Write-Host -ForegroundColor Yellow "Nadváha."
}
elseif ($bmi -le 35) {
  Write-Host -ForegroundColor Orange "Obezita prvního stupně."
}
elseif ($bmi -le 40) {
  Write-Host -ForegroundColor Red "Obezita druhého stupně."
}
else {
  Write-Host -ForegroundColor Magenta "Morbidní obezita."
}

\end{minted}
\end{center}

\framebreak
\textbf{Ternární operátor}
\begin{itemize}
  \item Nově s verzí 7.0
  \item Zápis \textit{If, Else} v jednom příkaze
  \item \texttt{<podmínka> ? <if-true> : <if-false>}
\end{itemize}
\vspace{5mm}
    \begin{center}
    \begin{minted}[escapeinside=||]{powershell}
$path = "/proc/self"
# Příkaz Test-Path zjišťuje,
# zda v systému existuje daná cesta.
$system = (Test-Path $path) ? "Unix" |:| "Windows"  # Unix
    \end{minted}
    \end{center}
 \framebreak
\textbf{Switch}
\begin{itemize}
  \item Větvení programu na základě hodnoty jedné proměnné
  \item Lze testovat přesné hodnoty, porovnávat pomocí operátorů nebo ověřovat vůči regulárnímu výrazu
  \item Dá se použít i na kolekce
  \item Možnost Default slouží jako výchozí, když porovnání se všemi ostatními větvemi neuspělo
\end{itemize}
\textbf{Syntax}
    \begin{minted}[mathescape]{powershell} 
Switch(<proměnná>) {
  <možnost1> {<příkazy>}
  <možnost2> {<příkazy>}
  # $\ldots$
  <možnostN> {<příkazy>}
  Default    {<příkazy>}
}
  \end{minted}
  \begin{itemize}
    \item Klíčová slova \textit{Break} a \textit{Continue} se hodí při testování hodnot pole nebo pokud Switch obsahuje více možností odpovídajících pro danou hodnotu
      \begin{itemize}
        \item Příkaz \textit{Break} ukončí testování aktuální i všech následujících hodnot kolekce.
        \item Příkaz \textit{Continue} ukončí testování pouze aktuální hodnoty.
      \end{itemize}
   \item Aktuální testovaná hodnota je uložená v proměnné \mintinline{powershell}|$_|
  \end{itemize}
\framebreak
\noindent \textbf{Příklad: Switch v poli}
\vspace{3mm}
  \begin{columns}
    \begin{column}{0.6\textwidth}
 \begin{minted}{powershell}
Switch (1,4,-1,3,"Číslo",2,1)
{
  1 { "Lednička" }
  {$_ -lt 0} { Continue }
  {$_ -isnot [Int32]} { Break }
  {$_ % 2} {
  "$_ je liché"
  }
  {-not ($_ % 2)} {
  "$_ je sudé"
  }
}     
 \end{minted}
    \end{column}
    \begin{column}{0.3\textwidth}
      \textbf{Výstup}
      \begin{verbatim}
Lednička
1 je liché
4 je sudé
3 je liché       
      \end{verbatim}

    \end{column}
  \end{columns}
\end{frame}

\subsection{Cykly}
\begin{frame}[allowframebreaks,fragile]{Cykly}
  \begin{itemize}
    \item Podmíněné opakování příkazů
    \item Stejně jako u switche lze použít \textit{Break} nebo {Continue} k předčasnému ukončení 
  \end{itemize}
  
  \textbf{Cyklus For}

  \begin{center}
      \begin{minted}{powershell}
  for (<Inicializace>; <Podmínka>; <Opakování>) {
    <příkazy>
  }
    \end{minted}
  \end{center}
  \begin{itemize}
    \item \texttt{<Inicializace>} umožňuje vytvořit iterační proměnné
    \item \texttt{<Podmínka>} je výraz, který musí být splněn, aby se cyklus mohl opakovat
      \begin{itemize}
        \item Kontrola proběhne také před prvním spuštěním cyklu
      \end{itemize}
    \item V \texttt{<Opakování>} jsou obsaženy všechny příkazy, které se provedou po skončení jedné iterace cyklu
      \item Žádná z těchto tří částí není vyžadována
  \end{itemize}
  \framebreak
  \begin{columns}[t]
    \begin{column}{0.6\textwidth}
      \textbf{Příklad: FizzBuzz}
    \begin{minted}[fontsize=\small]{powershell}
for ($i = 1; $i -le 20; $i++) {
  if ($i % 3 -eq 0 -and $i % 5 -eq 0) {
    Write-Host "FizzBuzz," -NoNewline
  }
  elseif ($i % 3 -eq 0) {
    Write-Host "Fizz," -NoNewline
  }
  elseif ($i % 5 -eq 0) {
    Write-Host "Buzz," -NoNewline
  }
  else {
    Write-Host "$i," -NoNewline
  }
}
# Výstup: 1,2,Fizz,4,Buzz,Fizz,7,8,
# Fizz,Buzz,11,Fizz,13,14,FizzBuzz,
# 16,17,Fizz,19,Buzz,
  \end{minted}
    \end{column}
    \begin{column}{0.4\textwidth}
     \textbf{Zadání}
     \begin{itemize}
       \item Výpis čísel od 1 do 20.
       \item Násobky 3 jsou nahrazeny slovem \mintinline{powershell}|"Fizz"|
       \item Násobky 5 jsou nahrazeny slovem \mintinline{powershell}|"Buzz"|
       \item Násobky 3 i 5 jsou nahrazeny \mintinline{powershell}|"FizzBuzz"|
       \item Pozn.: Příkaz \mintinline{powershell}|$i++| je ekvivalentem \mintinline{powershell}|$i = $i + 1| nebo~\mintinline{powershell}|$i += 1|
     \end{itemize}
    \end{column}
  \end{columns}
\framebreak
\begin{columns}
  \begin{column}{0.70\textwidth}
  \textbf{Příklad: Vnořené cykly\\Pascalův trojúhelník}
  \begin{minted}[fontsize=\scriptsize]{powershell}
$size = 5
$values = @(1)
for ($i = 1; $i -le $size; $i++) {
  $spaces = " " * ($size - $i)
  $row = ""
  $nextValues = @(1)
  for ($i = 0; $i -lt $values.Length; $i++) {
    # Přičtení prvku kolekce z i-té pozice
    $row += $values[$i]
    $row += " "
    if ($i + 1 -lt $values.Length) {
      $nextValues += $values[$i] + $values[$i + 1]
    }
  }
  Write-Host $spaces$row
  # Operátor += pro pole znamená
  # přidání prvku na jeho konec
  $nextValues += 1
  $values = $nextValues 
}
  \end{minted}
  \end{column}
  \begin{column}{0.215\textwidth}
    \begin{center}
    \textbf{Výstup}
    \begin{verbatim}
      1
     1 1
    1 2 1
   1 3 3 1
  1 4 6 4 1
    \end{verbatim}
    \end{center}
  \end{column}
\end{columns}
\framebreak
\textbf{Cyklus ForEach}
\begin{itemize}
  \item Průchod přes kolekci
\end{itemize}
  \begin{center}
      \begin{minted}{powershell}
  foreach ($<položka> in $<kolekce>) {
    <příkazy>
  }
    \end{minted}
  \end{center}
     \textbf{Příklad}
     \begin{itemize}
       \item Vypsání informací o všech souborech větších než 100 KB v aktuální složce
     \end{itemize}
    \begin{minted}{powershell}
  foreach ($soubor in Get-ChildItem) {
    if ($soubor.length -gt 100KB) {
      Write-Host $soubor
      Write-Host $soubor.Length
      Write-Host $soubor.LastAccessTime
    }
  }
     \end{minted}
 \framebreak
 \textbf{Cyklus While}
 \begin{itemize}
   \item Stejný jako cyklus For bez inicializace a příkazů po skončení
   \item Hodí se, když nepotřebujeme vytvářet další proměnné nebo nechceme provádět některé příkazy (třetí část For) po zavolání \textit{Continue}
 \end{itemize}
 \begin{center}
     \begin{minted}[mathescape]{powershell}
 <Inicializace>
 while (<Podmínka>) {
   <příkazy>
   # $\ldots$
   <Opakování>
 }
     \end{minted}
 \end{center}
 \framebreak
 \textbf{Příklad: největší společný dělitel dvou přirozených čísel\\(Euklidův algoritmus)}
     \begin{minted}{powershell}
[int]$a = Read-Host "Zadejte první číslo"
[int]$b = Read-Host "Zadejte druhé číslo"

while($b -ne 0){
  $rem = $a % $b
  $a = $b
  $b = $rem
}

Write-Host "Největší společný dělitel:" $a
     \end{minted}
\framebreak

\textbf{Cyklus Do}
\begin{itemize}
  \item Nejdříve vykoná příkazy, podmínky kontroluje až potom
  \item Dvě varianty: \textbf{Do While}, \textbf{Do Until} - podmínka je negovaná
  \item Vhodné při zpracování vstupu (nejdříve načíst, až pak ověřit)
\end{itemize}
\begin{center}
\begin{minted}[mathescape]{powershell}
  <Inicializace>
  do {
    <příkazy>
    # $\ldots$
    <Opakování>
  }
  while (<Podmínka>)
  \end{minted}
\end{center}
\framebreak
\textbf{Příklad: Hádání čísla}
  \begin{minted}[fontsize=\small]{powershell}
$target = Get-Random -Maximum 100  # Náhodné číslo mezi 0 a 99
$numTries = 0 

do {
  $numTries++
  [int]$guess = Read-Host "Zadejte číslo"
  if ($guess -lt $target) {
    Write-Host "Více."
  }
  elseif ($guess -gt $target) {
    Write-Host "Méně."
  }
} while ($guess -ne $target)

Write-Host "Trefa! Uhádli jste po $numTries pokusech"
\end{minted}
\end{frame}

\end{document}
