\documentclass[main.tex]{subfiles}

\begin{document}
\subsection{Pipeline}
\begin{frame}[fragile]{Pipeline}
  \begin{itemize}
    \item Řetězení několika příkazů na jednom řádku (\textit{one-linery})
    \item Předávání výstupů mezi příkazy bez nutnosti vytváření proměnných na mezivýsledky
    \item K aktuálnímu výstupu příkaz přistupuje ve speciálním (vyhrazeném) objektu \mintinline{powershell}|$_|
      \begin{itemize}
        \item V dokumentaci jednotlivých cmdletů je nutné dohledat, v jakých vlastnostech skončí daný vstup
      \end{itemize}
    \item Zapisuje se operátorem \textbf{|} (pipe, roura)
    \item Jestliže je výsledný pipeline moc dlouhý, doporučuje se rozdělit ho kvůli přehlednosti na více řádků končící (nebo začínající) rourou
      \begin{itemize}
        \item Umožňuje případné komentování jednotlivých částí
      \end{itemize}
  \end{itemize}
\vspace{2mm}
\begin{columns}
  \begin{column}{0.40\textwidth}
    \textbf{Použití s pipeline}
    \begin{minted}[fontsize=\scriptsize]{powershell}
Get-Process |
Sort-Object Name |
Format-Table -Property Name, CPU
    \end{minted}
  \end{column}
  \begin{column}{0.50\textwidth}
    \textbf{Použití bez pipeline}
    \begin{minted}[fontsize=\scriptsize]{powershell}
$Processes = Get-Process
$SortedPS = Sort-Object Name
Format-Table $SortedPS -Property Name, CPU
  \end{minted}
  \end{column}
\end{columns}
\end{frame}

\begin{frame}[fragile,allowframebreaks]{Filtrování v pipeline}
\textbf{Where-Object}
\begin{itemize}
  \item Výběr prvků v kolekci, jejichž vlastnosti splňují danou podmínku
  \item Kromě ScriptBlocku lze také porovnávat přímo v parametrech cmdletu
  \item Často se používá alias \texttt{where}
  \end{itemize}
  Příklad: \textit{Výběr procesů s nízkou prioritou} (ekvivalentní způsoby)
   \begin{enumerate}\scriptsize
   \item \noindent\mintinline{powershell}@Get-Process | Where-Object PriorityClass -eq -Value "BelowNormal"@
      \item \mintinline{powershell}@Get-Process | Where-Object -Property PriorityClass -eq "BelowNormal"@
      \item \mintinline{powershell}@Get-Process | where {$_.PriorityClass -eq "BelowNormal"}@
    \end{enumerate}
 \begin{itemize}
   \item Dá se použít také na ověření existence dané vlastnosti
 \end{itemize}
  Příklad: \textit{Výběr všech adresářů v aktuální složce}
  \mintinline{powershell}@Get-ChildItem | where {$_.PSIsContainer}@

\framebreak
  \textbf{Select-Object}\\[2mm]
  Funkce:
    \begin{itemize}
       \setlength\itemsep{1em}
       \item Výběr vlastností v každém prvku: \textbf{\texttt{-Property}}
         {\small
        \mintinline{powershell}@Get-Process | Select-Object -Property ProcessName, Id, WS@
        }
      \item Filtrování jedinečných hodnot: \textbf{\texttt{-Unique}}
        \begin{minted}{powershell}
"a","b","c","a","c","a" | Select-Object -Unique
# výstup: a b c
        \end{minted}
      \item Upřesnění rozmezí:
        \begin{itemize}
          \item jednotlivé pozice (\textbf{\texttt{-Index}} \texttt{<Seznam>})
        \begin{minted}{powershell}
$a = Get-EventLog -LogName "Windows PowerShell"
# výběr prvního a posledního záznamu
$a | Select-Object -Index 0, ($a.count - 1)
        \end{minted}
        \framebreak
          \item prvních (\textbf{\texttt{-First}}) \textit{N} prvků
            \begin{minted}{powershell}
# Prvních 5 řádků v souboru
Get-Content file.txt | select -First 5
            \end{minted}
          \item posledních (\textbf{\texttt{-Last}}) \textit{N} prvků
            \begin{minted}{powershell}
# Poslední 3 řádky ze schránky
Get-Clipboard | select -Last 3
            \end{minted}
          \item přeskočení \textit{N} prvků (\textbf{\texttt{-Skip,-SkipLast}}) nebo vybraných pozic (\textbf{\texttt{-SkipIndex}})
\begin{minted}{powershell}
# Vytvoření vzdáleného sezení na všech serverech
# kromě prvního serveru v souboru
New-PSSession -ComputerName (Get-Content Servers.txt |
Select-Object -Skip 1)
\end{minted}
        \end{itemize}
      \item Pozn.: Při použití \texttt{-First} nebo \texttt{-Index} se vstup přestane generovat (nebo zpracovávat) ve chvíli, kdy se načtou všechny požadované prvky
        \begin{itemize}
          \item K vypnutí této optimalizace je třeba upřesnit přepínač \texttt{-Wait}
        \end{itemize}
    \end{itemize}
\framebreak
\textbf{Sort-Object}
\begin{itemize}
  \item Setřízení kolekce na základě vlastností
  \item Výchozím parametrem řazení je atribut \textbf{\texttt{Name}}
  \item Lze upřesnit získání pouze prvních (\textbf{\texttt{-Top}}) nebo posledních (\textbf{\texttt{-Bottom}}) \textit{N} prvků z výsledné struktury\vspace{2mm}
    \begin{minted}[fontsize=\footnotesize]{ps1}
Get-ChildItem "~/.config" | Sort-Object -Top 5
<#
Directory: /home/auburn/.config

  Mode                 LastWriteTime         Length Name
  ----                 -------------         ------ ----
  d----           5/23/2020  3:23 PM                alacritty
  d----            5/9/2020 11:03 PM                asciinema
  d----           3/28/2020  3:00 PM                autostart
  d----            5/3/2020  2:24 PM                blender
  d----           4/21/2020 11:03 PM                calcurse
#>
    \end{minted}
  \item Prvky lze řadit buď vzestupně (výchozí způsob), nebo sestupně (\textbf{\texttt{-Descending}})\vspace{3mm}
    \begin{minted}[fontsize=\footnotesize]{powershell}
Get-History | Sort-Object -Property Id -Descending
<#
  Id CommandLine
  -- -----------
  10 Get-Command Sort-Object -Syntax
   9 $PSVersionTable
   8 Get-Command Sort-Object -Syntax
   7 Get-Command Sort-Object -ShowCommandInfo
   6 Get-ChildItem -Path ~/Pictures | Sort-Object -Property Length
   5 Get-Help Clear-History -online
   4 Get-Help Clear-History -full
   3 Get-ChildItem | Get-Member
   2 Get-Command Sort-Object -Syntax
   1 Set-Location ~/Documents
#>
  \end{minted}
\framebreak
  \item Opět možno vybírat pouze jedinečné hodnoty přepínačem \textbf{\texttt{-Unique}}
  \item Parametr \textbf{\texttt{-Stable}} zaručí, že rovnající se hodnoty budou ve výsledku mít stejné pořadí jako v původní struktuře
  \item Vlastní styl porovnávání se dá definovat ve ScriptBlocku (výchozí porovnává na základě typu vlastností)
\end{itemize}
\framebreak
\textbf{Příklad: Řazení na základě zbytku po dělení}
\begin{columns}[t]
  \begin{column}{0.47\textwidth}
    \textbf{Výchozí třídění}
  \scriptsize
    \begin{minted}{powershell}
1..17 | Sort-Object {$_ % 3}
9
15
3
12
6
13
10
16
1
7
4
11
5
14
2
8
17
\end{minted}
  \end{column}
  \begin{column}{0.5\textwidth}
    \textbf{Stabilní třídění}
  \scriptsize
    \begin{minted}{powershell}
1..17 | Sort-Object {$_ % 3} -Stable
3
6
9
12
15
1
4
7
10
13
16
2
5
8
11
14
17
\end{minted}
  \end{column}
\end{columns}
\end{frame}

\subsection{Funkce}
\begin{frame}[allowframebreaks,fragile]{Funkce}
  \begin{itemize}
    \item Samostatný blok kódu
      \begin{itemize}
        \item Po definici se nespustí sám, je třeba ho zavolat
      \end{itemize}
    \item Název funkce by měl dodržovat standardní formát Cmdletu
    \item Lze ji zadefinovat s parametry, případně v ní vracet hodnotu pro použití v jiných příkazech nebo předčasné skončení
    \item Členění příkazů na menší logické celky pomůže zvýšit přehlednost a~čitelnost kódu
    \item Snižuje potřebu kopírování kódu
  \end{itemize}
  \framebreak
\textbf{Základní syntax}
  \begin{minted}{powershell}
function <název-funkce> {<příkazy>}
\end{minted}
\textbf{Příklad}
\begin{minted}{powershell}
function Get-Weekday { (Get-Date).DayOfWeek }
\end{minted}
\vspace{3mm}
\begin{itemize}
  \item Část funkce v \texttt{\{ \}} se nazývá \textbf{ScriptBlock} (typ)
  \item Lze ji využít anonymně (bez názvu) v pipelinech
\end{itemize}
\textbf{Příklad: ScriptBlock u WhereObject}
  \begin{minted}[escapeinside=@@]{powershell}
    Get-ChildItem -Path *.txt | where @\colorbox{yellow!80}{\texttt{\textbf{\{\$\_.length -gt 10000\}}}}@
\end{minted}
\framebreak
\textbf{Parametry}
\begin{itemize}
  \item Proměnné vnášené do funkce
\end{itemize}
  \textit{Pojmenované parametry}\\[2mm]
  \mintinline{powershell}|Get-ChildItem -Path "~/.scripts" -Depth 2|

\begin{itemize}
  \item Jsou zadány jménem, popř.\ typem
  \item Powershell umožňuje odkazovat se na ně jejich částečným názvem (mimojiné podporuje doplňování tabulátorem), pokud je ovšem jednoznačný
    \begin{itemize}
      \item Vhodné pouze pro rychlé testování v konzoli
    \end{itemize}
\end{itemize}
\framebreak

{\large\textbf{Příklad: částečné názvy}}

\vspace{3mm}
 \textbf{Nejednoznačný název}\\
 \mintinline{ps1con}|PS /home/auburn> Get-ChildItem -P "~/.scripts"|

{\color{red}\texttt{Get-ChildItem: Parameter cannot be processed because the parameter name 'P' is ambiguous. Possible matches include: -Path -PipelineVariable -LiteralPath.}}\\

 \textbf{Jednoznačný název}
\begin{minted}{ps1con}
PS /home/auburn> Get-ChildItem -Pa "~/.scripts"

    Directory: /home/auburn/.scripts

    Mode                 LastWriteTime         Length Name
    ----                 -------------         ------ ----
    -----            5/6/2020  1:29 AM           1165 btw,
\end{minted}
\framebreak

\begin{itemize}
  \item Možnost přiřadit také výchozí hodnotu (využije se, když nebude při volání funkce určena)
  \item Dva způsoby zápisu (ekvivalentní, ale styl vlevo se v Powershellu používá více)
\end{itemize}
\begin{columns}[t]
  \begin{column}{0.4\textwidth}
    \small\textbf{Zápis ve ScriptBlocku s \texttt{param}}
\begin{minted}[fontsize=\footnotesize]{powershell}
function <název> {
  param([typ]$parametr1,
        [typ]$parametr2,
        ...,
        [typ]$parametrN)

  <příkazy>
}
\end{minted}
 \end{column}
  \begin{column}{0.5\textwidth}
    \footnotesize\textbf{Zápis podobný u běžných prog. jazyků \\(C++, Java)}
\begin{minted}[fontsize=\footnotesize]{powershell}
function <název> ([typ]$parametr1,
                  [typ]$parametr2,
                  ...,
                  [typ]$parametrN)
  {
    <příkazy>
  }
\end{minted}
 \end{column}
\end{columns}
\framebreak
\textbf{Příklad: Funkce s výchozí hodnotou\\Soubory menší než \texttt{Size}}

\begin{minted}[fontsize=\footnotesize]{powershell}
function Get-SmallFiles {
  param (
    $Size = 100,
    $Dir = $HOME
  )
  Get-ChildItem $Dir | Where-Object {
    # všechny objekty menší než $Size, které nejsou složkami
    $_.Length -lt $Size -and !$_.PSIsContainer
  }
}
\end{minted}
\begin{itemize}
  \item Lze zavolat např.\
\begin{minted}[fontsize=\footnotesize]{powershell}
Get-SmallFiles -Dir "." # aktuální složka
# ~ je zkratka pro domovský adresář uživatele na Linuxu
Get-SmallFiles -Dir "~/Pictures" -Size 10000
# nejmenované parametry musí být ve stejném pořadí jako v param
Get-SmallFiles 1024 "~/.config"
  \end{minted}
\end{itemize}
\framebreak
\textit{Poziční parametry}
\begin{itemize}
  \item Nepoužívají se tak často jako pojmenované parametry, protože ty lze v případě potřeby na poziční snadno převést
  \item Do funkce se vnáší pouze beze jména v pevně daném pořadí (ale mohou být prokládány jmenovanými parametry, čímž se volání může stát nepřehledným)
  \item Uložené v proměnné \mintinline{powershell}|$args| (pole)
\end{itemize}

\textbf{Příklad: Zápis do souboru ze schránky}
  \begin{minted}{powershell}
function Write-Clipboard {
  Get-Clipboard | Set-Content -Path $args[0]
}
# Použití:
Write-Clipboard vystup.out
  \end{minted}
\framebreak
\textit{Povinné parametry}
\begin{itemize}
  \item Upřesňují se v atributu \texttt{Parameter}
    \begin{itemize}
      \item Zde se dá nastavit také pozice samotného parametru, čímž se jmenovaný parametr zkombinuje s pozičním
    \end{itemize}
  \item Jestliže se funkce zavolá bez nich, bude se vyžadovat jejich doplnění (interaktivně od uživatele)
\end{itemize}
\textbf{Příklad: Vyhledávání souboru}
\begin{minted}{powershell}
function Find-File {
  param(
    $Dir, # nepovinný jmenovaný parametr
    [Parameter(Mandatory, Position=0)]
    [string]$Name # povinný poziční parametr
  )
  Get-ChildItem $Dir -Recurse -Filter $Name -File
}
\end{minted}
\framebreak

\textit{Přepínače}
\begin{itemize}
  \item Jmenované parametry s hodnotami typu \textit{Boolean} (\texttt{\$true,\$false})
  \item K přiřazení hodnoty \texttt{\$true} stačí při volání zapsat pouze jejich název
  \item Definují se typem \texttt{[switch]}
\end{itemize}
\begin{columns}
  \begin{column}{0.38\textwidth}
 \textbf{Příklad}
  \begin{minted}[fontsize=\small]{powershell}
function Write-Clipboard {
  param(
    [string]$Path,
    [switch]$Debug
  )
  if ($Debug) {
    Get-Clipboard
  }
  Get-Clipboard |
  Set-Content -Path $Path
}
\end{minted}
  \end{column}
  \begin{column}{0.62\textwidth}
    \textbf{Volání pro \texttt{Debug=\$true}}
\begin{minted}[fontsize=\small,escapeinside=||]{powershell}
Write-Clipboard -Debug -Path a.out
Write-Clipboard -Debug|:|$true -Path a.out
  \end{minted}

\textbf{Volání pro \texttt{Debug=\$false}}
\begin{minted}[fontsize=\small,escapeinside=||]{powershell}
Write-Clipboard -Path a.out
Write-Clipboard -Debug|:|$false -Path a.out
  \end{minted}

  \end{column}
\end{columns}

\framebreak
\end{frame}

\begin{frame}[fragile,allowframebreaks]{Pipeline ve funkcích}
  \vspace{-5mm}
  \begin{itemize}
  \item Pokročilejší ovládání vstupu z pipeline
  \item Využití při průchodu přes kolekci\\[3mm]
\end{itemize}

\textbf{Syntax}
  \begin{minted}{powershell}
function <název> {
  begin {<příkazy>}
  process {<příkazy>}
  end {<příkazy>}
}
\end{minted}
\begin{itemize}
  \item Části \textit{Begin, End} se spouští pouze jednou - při první či poslední zpracované položce
  \item Část \textit{Process} proběhne pro každou položku kolekce
    \framebreak
  \item Během volání funkce je dostupná speciální proměnná \texttt{\$input}, ve~které je uložen dosud zpracovaný vstup
    \begin{itemize}
      \item V \textit{Begin} tedy bude prázdná, a v \textit{End} bude obsahovat původní objekt poslaný přes pipeline
      \item Jestliže je definován blok \textit{Process}, hodnoty se přesouvají do proměnné \texttt{\$\_}, a \texttt{\$input} tak bude vždy prázdný
      \item Lze ji použít i mimo bloky, kde její hodnota bude stejná jako po přečtení v \textit{End}
    \end{itemize}
  \item Pokud je definován jakýkoliv z těchto bloků, veškerý kód mimo ně se už nebere jako kód a při zavolání funkce spadne (s dost kryptickými chybami)
\end{itemize}
\vspace{-7mm}
\begin{columns}[t]
  \begin{column}{0.50\textwidth}
 \begin{minted}[fontsize=\scriptsize,bgcolor=white]{powershell}
function Get-PipelineInput {
  process {"Aktuální objekt: $_ "}
  end {"Konec:   Celý vstup: $input"}
}
\end{minted}
\end{column}
  \begin{column}{0.53\textwidth}
\begin{minted}[fontsize=\scriptsize,bgcolor=bg1]{ps1con}
PS /home/auburn> 1,2,4 | Get-PipelineInput
Aktuální objekt: 1
Aktuální objekt: 2
Aktuální objekt: 4
Konec: Celý vstup:
\end{minted}
  \end{column}
\end{columns}
\framebreak
\textbf{Filtr}
\begin{itemize}
  \item Ekvivalent funkce, jejíž kód je celý zabalený do bloku \textit{Process}
  \item Definuje se klíčovým slovem \texttt{filter} na místě \texttt{function}
\end{itemize}
\textbf{Příklad}
\begin{itemize}
  \item Filtr, která zobrazuje buď celý záznam, nebo pouze zprávu (člena \texttt{Message}) z výpisu
\end{itemize}
\begin{minted}{powershell}
filter Get-ErrorLog ([switch]$message)
{
  if ($message) { Out-Host -InputObject $_.Message }
  else { $_ }
}
\end{minted}
\end{frame}
\end{document}
