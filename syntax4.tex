\documentclass[main.tex]{subfiles}

\begin{document}
\subsection{Objekty}%
\begin{frame}[fragile,allowframebreaks]{Objekty}
Složitější datové struktury umožňující definici vlastních typů.\\

  \textbf{Pojmy}
  \begin{itemize}
    \item \textit{Třída} (Class) - definice objektu
    \item \textit{Instance} - konkrétní proměnná s typem tohoto objektu
    \item \textit{Atribut} (Property) - proměnná uvnitř objektu
    \item \textit{Metoda} - funkce definovaná pro objekt, má přístup k jeho atributům
      \begin{itemize}
        \item Možnost definovat více metod se stejným názvem lišící se v seznamu jejich parametrů (\textit{Method Overloading} - přetěžování metod)
      \end{itemize}
    \item \texttt{\$this} - reference na objekt uvnitř jeho metod
    \item \textit{Konstruktor} - metoda volaná při vzniku objektu, slouží k inicializaci jeho atributů
    \item \textit{Statický člen} - Atribut nebo metoda existující v rámci třídy jako takové, nepatří žádné konkrétní instanci
  \end{itemize}
\framebreak
\begin{columns}[t]
  \begin{column}{0.35\textwidth}
\textbf{Příklad: vytvoření třídy}
    \begin{minted}[fontsize=\fontsize{8.3pt}{8.7pt},linenos=true]{powershell}
class Tea {
  # statický atribut:
  # celkový počet nálevů
  static [int]$TotalInfusions = 0
  # atributy každé instance
  [string]$Name
  [string]$Origin
  [int]$Infusions
  # konstruktory objektu
  Tea($Name, $Origin, $Infusions){
    $this.Name      = $Name
    $this.Origin    = $Origin
    $this.Infusions = $Infusions
  }
  Tea(){ $this.Name = 'Undefined' }
  # metoda v instanci
  [void] Pour(){
    if($this.Infusions -gt 0){
      $this.Infusions--
      # přístup ke statickému atributu
      [Tea]::TotalInfusions++
    }
  }
}
\end{minted}
\end{column}
\begin{column}{0.5\textwidth}
  \small\textbf{Viditelnost atributů}
  \begin{itemize}
    \item Veškeré atributy třídy jsou \textbf{veřejné}
      \begin{itemize}
        \item Jestliže máme odkaz na instanci, můžeme v ní číst, měnit a používat úplně všechno
        \item Pokud víme o existenci objektu, dostaneme se i ke všem jeho statickým atributům
        \item Členové s vlastností \textbf{hidden} (nastavuje se podobně jako \texttt{static}) se pouze nezobrazují při doplňování tabulátorem nebo při zavolání \texttt{Get-Member} bez přepínače \texttt{-Force}
      \end{itemize}
  \end{itemize}
\end{column}
\end{columns}
\framebreak
\textbf{Příklad: čajový dýchánek}
\begin{center}
\begin{minted}[linenos=true,fontsize=\footnotesize]{powershell}
# 1. způsob nastavení atributů:
# Upřesnění v konstruktoru při instancování
$cj = [Tea]::new('China Jasmin', 'China', 3)
$cj.Pour()
$cj.Pour()
# 2. způsob: Přepsání po vytvoření
$Mate = [Tea]::new()
$Mate.Name = 'Mate Atacama'
$Mate.Origin = 'Brazil'
$Mate.Infusions = 1
$Mate.Pour()
# 3. způsob: Cmdlet New-Object
$Rooibos = New-Object Tea -ArgumentList "Rooibos Pretoria",
           "South Africa", 1
$Rooibos.Pour()
$Rooibos.Pour()
[Tea]::TotalInfusions  # 4
\end{minted}
\end{center}
\framebreak
\textbf{Dědičnost}
\begin{itemize}
  \item Rozšíření vlastností objektu v další třídě
  \item Třída, která dědí, má automaticky přístup ke všem členům třídy původní
  \item Atributy rodičovské třídy se inicializují v \texttt{base()}
  \item Zděděné metody se mohou dle potřeby přepisovat
\end{itemize}
\framebreak
\textbf{Příklad\\Základová třída}
\begin{minted}[escapeinside=||,fontsize=\small]{powershell}
class Beverage {
  [double]$Volume
  [double]$Price
  [void] Pour() {
    $this.Volume /= 2
  }

  Beverage([double]$Volume, [double]$Price){
    $this.Volume = $Volume
    $this.Price  = $Price
  }
}
\end{minted}
\framebreak
\textbf{Příklad\\Dědící třída}
\begin{minted}[escapeinside=||,fontsize=\scriptsize]{powershell}
class Juice |:| Beverage {
  [double]$FruitPart
  Juice($Volume, $Price, $FruitPart) |:| base($Volume, $Price) {
    $this.FruitPart = $FruitPart
  }
  # Přetížená metoda
  [void] Pour() {
    $this.Volume /= 10
  }
  [string]ToString(){
    return ("Juice {0}% - ${1} - {2} ml" -f $this.FruitPart,
            $this.Price, $this.Volume)
  }
}
$caprisonne = [Juice]::new(250, 0.5, 10)
$caprisonne.Pour()
$caprisonne.ToString()
# Juice 10% - $0.5 - 25 ml
\end{minted}
\end{frame}
\end{document}
